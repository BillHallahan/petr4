\documentclass[10pt]{article}
\usepackage{mathpartir}
\usepackage{amsmath}
\usepackage{amssymb}

\usepackage{xifthen}

\usepackage{hyperref}

\usepackage{tgpagella}
\usepackage{mathpazo}

\newcommand{\todo}{\textsf{TODO}}

\newcommand{\Sequent}[2]{\ensuremath{#1 \vdash #2}}
\newcommand{\DoubleSequent}[3]{\ensuremath{#1 \vdash #2 \dashv #3}}
\newcommand{\Numeric}[1]{\ensuremath{\operatorname{numeric}(#1)}}
\newcommand{\Keyword}[1]{\ensuremath{\mathsf{#1}}}

% Macros for BNF-ish abstract syntax
\newcommand{\SynClassAndCase}[2]{#1 &\mathbin{::=} & #2 \\}
\newcommand{\SynCase}[1]{& \mathbin{|} & #1 \\}
\newenvironment{Syntax}
  {
      $$
      \begin{array}{lcl}
  }
  {
      \end{array}
      $$
  }

% macros for Petr4 syntax
\newcommand{\Error}{\Keyword{error}}
\newcommand{\MatchKind}{\Keyword{match\_kind}}
\newcommand{\Void}{\Keyword{void}}
\newcommand{\Bool}{\Keyword{bool}}
\newcommand{\True}{\Keyword{true}}
\newcommand{\False}{\Keyword{false}}
\newcommand{\Varbit}[1]{\Keyword{varbit}\langle#1\rangle}
\newcommand{\Int}[1]{
    \Keyword{int}
    \ifthenelse{\equal{#1}{}}
        {}
        {\langle #1 \rangle}
}
\newcommand{\Bit}[1]{\Keyword{bit}\langle#1\rangle}
\newcommand{\Stack}[2]{#1\,\Keyword{stack}\langle#2\rangle}
\newcommand{\Set}[1]{#1\,\Keyword{set}}
\newcommand{\Prod}{\times}
\newcommand{\Arrow}{\longrightarrow}
\newcommand{\TApp}[2]{#1 \langle #2 \rangle}
\newcommand{\SignedLit}[2]{#1\Keyword{s}#2}
\newcommand{\UnsignedLit}[2]{#1\Keyword{u}#2}
\newcommand{\Member}{.}
\newcommand{\Index}[2]{#1 [#2]}
\newcommand{\Slice}[3]{#1 [#2 : #3]}
\newcommand{\SetLit}[1]{\{#1\}}
\newcommand{\Many}[1]{\overline{#1}}
\newcommand{\Uop}{\ominus}
\newcommand{\Binop}{\odot}
\newcommand{\Cast}[2]{(#1)#2}
\newcommand{\Ternary}[3]{#1 \mathbin{?} #2 \mathbin{:} #3}
\newcommand{\AppWithTypes}[3]{
    #1
    \ifthenelse{\equal{#2}{}}
        {}
        {\langle #2 \rangle}
    (#3)
}
\newcommand{\FunApp}[3]{\AppWithTypes{#1}{#2}{#3}}
\newcommand{\Mask}{\mathbin{\&\&\&}}
\newcommand{\Range}{\mathbin{..}}
\newcommand{\Header}[2]{\Keyword{header}\ #1\ \{ #2 \}}
\newcommand{\HeaderUnion}[2]{\Keyword{header\_union}\ #1\ \{ #2 \}}
\newcommand{\Struct}[2]{\Keyword{struct}\ #1\ \{ #2 \}}
\newcommand{\ErrorDecl}[1]{\Error.f : \Error}
\newcommand{\Enum}[2]{\Keyword{enum}\ #1\ \{ #2 \}}
\newcommand{\SEnum}[2]{\Keyword{enum}\ #1\ \{ #2 \}}
\newcommand{\Extern}[3]{
    \Keyword{extern}\ #1
    \ifthenelse{\equal{#2}{}}
        {}
        {\langle #2 \rangle}\ \{ #3 \}
}
\newcommand{\Newtype}[2]{\Keyword{newtype}\ #1\ #2}
\newcommand{\Typedef}[2]{\Keyword{typedef}\ #1\ #2}
\newcommand{\Control}[3]{\Keyword{control}\ \AppWithTypes{#1}{#2}{#3}}
\newcommand{\Package}[3]{\Keyword{package}\ \AppWithTypes{#1}{#2}{#3}}
\newcommand{\Parser}[3]{\Keyword{parser}\ \AppWithTypes{#1}{#2}{#3}}
\newcommand{\Function}[3]{\Keyword{function}\ \AppWithTypes{#1}{#2}{#3}}
\newcommand{\ExternFunction}[3]{\Keyword{extern}\ \Keyword{function}\ \AppWithTypes{#1}{#2}{#3}}
\newcommand{\ValueSet}[3]{\AppWithTypes{\Keyword{value\_set}}{#1}{#2}\ #3}
\newcommand{\Action}[2]{\Keyword{action}\ \AppWithTypes{#1}{}{#2}}
\newcommand{\Table}[1]{\Keyword{table}\ #1\ \todo}

\newcommand{\Typed}[5]{\Sequent{#1;#2;#3}{#4 : #5}}
\newcommand{\TypedOut}[6]{\DoubleSequent{#1;#2;#3}{#4 : #5}{#6}}
\newcommand{\Var}{\Keyword{Var}}
\newcommand{\Str}{\Keyword{Str}}
\newcommand{\Name}{\Keyword{Name}}
\newcommand{\Z}{\mathbb{Z}}

\newcommand{\Exit}{\Keyword{exit}}
\newcommand{\Return}[1]{\Keyword{return}\ #1}
\newcommand{\Switch}[2]{\Keyword{switch}\ #1\ \{ #2 \}}
\newcommand{\If}[3]{\Keyword{if}\ (#1)\ #2\ \Keyword{else}\ #3}
\newcommand{\Nop}{\Keyword{nop}}

\newcommand{\Lookup}[2]{\operatorname{\Keyword{lookup}} #1\,#2}

\newcommand{\Spec}{https://p4.org/p4-spec/docs/P4-16-v1.1.0-spec.html}
\newcommand{\SpecRef}[1]{
    \footnote{\url{\Spec\##1}}
}

\begin{document}

Variables \(x, y \in \Var\).

Type names \(T \in \Name\). Both type variables and names of things like headers
are represented by type names.

Field names \(f \in \Name\).

Integer literals \(m, n \in \Z\).

Unary operators \(\ominus \in \{\dots\}\).

Binary operators \(\odot \in \{\dots\}\).

Directions for function arguments.
\begin{Syntax}
    \SynClassAndCase{p}{\Keyword{in}}
    \SynCase{\Keyword{out}}
    \SynCase{\Keyword{inout}}
\end{Syntax}

Expressions.
\begin{Syntax}
    \SynClassAndCase{e}{\True}
    \SynCase{\False}
    \SynCase{n}
    \SynCase{\SignedLit{m}{n}}
    \SynCase{\UnsignedLit{m}{n}}
    \SynCase{x}
    \SynCase{\Member f}
    \SynCase{\Index{e_1}{e_2}}
    \SynCase{\Slice{e_1}{e_2}{e_3}}
    \SynCase{\SetLit{\Many{e}}}
    \SynCase{\Uop e}
    \SynCase{e_1 \Binop e_2}
    \SynCase{\Cast{\tau}{e}}
    \SynCase{\tau \Member f}
    \SynCase{\Error \Member f}
    \SynCase{\Ternary{e_1}{e_2}{e_3}}
    \SynCase{\FunApp{e}{\Many{\tau}}{\Many{e}}}
    \SynCase{\FunApp{e}{}{\Many{e}}}
    \SynCase{e_1 \Mask e_2}
    \SynCase{e_1 \Range e_2}
\end{Syntax}

Blocks and statements.
\begin{Syntax}
    \SynClassAndCase{B}{\{ \overline{s} \}}
    \SynClassAndCase{s}{B}
    \SynCase{D}
    \SynCase{\FunApp{e}{\Many{\tau}}{\overline{x : \tau\,d}}}
    \SynCase{x = e}
    \SynCase{\FunApp{\tau}{}{\overline{x : \tau\,d}}}
    \SynCase{\If{e}{s}{s}}
    \SynCase{\Exit}
    \SynCase{\Nop}
    \SynCase{\Return{e}}
    \SynCase{\Switch{e}{\todo}}
\end{Syntax}

Declarations.
\begin{Syntax}
    \SynClassAndCase{D}{\Keyword{const}\ x : \tau = e}
    \SynCase{x : \tau = e}
    \SynCase{\tau(\Many{e})\,x}
    \SynCase{\Header{T}{\Many{f: \tau}}}
    \SynCase{\HeaderUnion{T}{\Many{f: \tau}}}
    \SynCase{\Struct{T}{\Many{f: \tau}}}
    \SynCase{\ErrorDecl{T}}
    \SynCase{\Enum{T}{\Many{f}}}
    \SynCase{\SEnum{T}{\Many{f: e}}}
    \SynCase{\Newtype{T}{\tau}}
    \SynCase{\Typedef{T}{\tau}}
    \SynCase{\Package{T}{\Many{T}}{\Many{x : \tau\,p}}}
    \SynCase{\Control{T}{\Many{T}}{\Many{x : \tau\,p}}}
    \SynCase{\Control{T}{}{\Many{x : \tau\,p}}(\Many{x : \tau})\ \{
        \overline{D}\ \Keyword{apply}\ B\}}
    \SynCase{\Action{f}{\Many{x : \tau\,p}, \Many{x : \tau}} B}
    \SynCase{\Parser{T}{\Many{T}}{\Many{x : \tau\,p}}}
    \SynCase{\Parser{T}{\Many{T}}{\Many{x : \tau\,p}}(\Many{x : \tau})\ \{ \todo \}}
    \SynCase{\ValueSet{\tau}{e}{T}}
    \SynCase{\Function{T}{\Many{T}}{\Many{x : \tau\,p}}\ \{ \todo \}}
    \SynCase{\ExternFunction{T}{\Many{T}}{\Many{x : \tau\,p}}\ \{ \todo \}}
    \SynCase{\Extern{T}{\Many{T}}{\todo}}
    \SynCase{\Table{T}}
\end{Syntax}

Types.
\begin{Syntax}
    \SynClassAndCase{\tau}{\Header{T}{\Many{f: \tau}}}
    \SynCase{\tau_1 \Prod \cdots \Prod \tau_n}
    \SynCase{(\Many{x: \tau\,d}) \to \tau'}
    \SynCase{(\Many{x: \tau}) \to \tau'}
    \SynCase{\Set{\tau}}
    \SynCase{\HeaderUnion{T}{\Many{f: \tau}}}
    \SynCase{\Stack{T}{n}}
    \SynCase{\Struct{T}{\Many{f: \tau}}}
    \SynCase{\ErrorDecl{T}}
    \SynCase{\Enum{T}{\Many{f}}}
    \SynCase{\SEnum{T}{\Many{f: e}}}
    \SynCase{\Extern{T}{\Many{T}}{\Many{m}}}
    \SynCase{\Newtype{T}{\tau}}
    \SynCase{\Control{T}{\Many{T}}{\Many{x : \tau\,p}}}
    \SynCase{\Parser{T}{\Many{T}}{\Many{x : \tau\,p}}}
    \SynCase{\Package{T}{\Many{T}}{\Many{x : \tau}}}
\end{Syntax}

Base types.
\begin{Syntax}
    \SynClassAndCase{\sigma}{\Void}
    \SynCase{\Error}
    \SynCase{\Bool}
    \SynCase{\Int{}}
    \SynCase{\Int{n}}
    \SynCase{\Bit{n}}
    \SynCase{\Varbit{n}}
\end{Syntax}

\begin{Syntax}
    \SynClassAndCase{\hat{\tau}}{.T}
    \SynCase{T}
    \SynCase{\alpha}
    \SynCase{\TApp{T}{\Many{\hat{\tau}}}}
\end{Syntax}


Type contexts \(\Delta = T_1: \tau_1, \dots, T_n: \tau_n\) where all \(T_i\) are
distinct.
Declaration contexts \(\Xi = D_1, D_2, \dots, D_n\).
Contexts \(\Gamma = x_1 : \tau_1, \dots, x_n: \tau_n\).

Typing judgements for expressions. \[
    \Typed{\Xi}{\Delta}{\Gamma}{e}{\tau}
\]

Rules for expression typing: booleans. \todo{} ternary operator.
\[
\begin{array}{c}
    \inferrule{{}}{\Typed{\Xi}{\Delta}{\Gamma}{\True}{\Bool}} \quad
    \inferrule{{}}{\Typed{\Xi}{\Delta}{\Gamma}{\False}{\Bool}} \\
    \inferrule{\Uop \in \{!\} \\ \Typed{\Xi}{\Delta}{\Gamma}{e}{\Bool}}
              {\Typed{\Xi}{\Delta}{\Gamma}{\Uop e}{\Bool}} \\
    \inferrule{\Binop \in \{\Keyword{\&\&}, \Keyword{||}\} \\
               \Typed{\Xi}{\Delta}{\Gamma}{e_1}{\Bool} \\
               \Typed{\Xi}{\Delta}{\Gamma}{e_2}{\Bool}}
              {\Typed{\Xi}{\Delta}{\Gamma}{e_1 \Binop e_2}{\Bool}}
\end{array}
\]
Rules for expression typing: integers. \todo: operations on integers.
\[
\begin{array}{c}
    \inferrule{{}}{\Typed{\Xi}{\Delta}{\Gamma}{n}{\Int{}}} \quad
    \inferrule{m > 1}{\Typed{\Xi}{\Delta}{\Gamma}{\SignedLit{m}{n}}{\Int{m}}}
    \quad
    \inferrule{{}}{\Typed{\Xi}{\Delta}{\Gamma}{\UnsignedLit{m}{n}}{\Bit{m}}}
\end{array}
\]
Rules for expression typing: variables.
\[
    \inferrule{\Lookup{\Gamma}{x} = \tau}{\Typed{\Xi}{\Delta}{\Gamma}{x}{\tau}}
\]

Rules for expresion typing: array operations. \todo

Rules for expresion typing: sets. \todo

Rules for expresion typing: casts. \todo

Rules for expresion typing: errors. \todo

Rules for expresion typing: set operations. \todo

Rules for expresion typing: applications.
\[
    \inferrule{\Typed{\Xi}{\Delta}{\Gamma}{e}{(\Many{\tau\,p}) \to \tau'} \\
    e_i : \tau_i \\
    \Keyword{dir}(e_i) \sqsubseteq p_i}
              {\Typed{\Xi}{\Delta}{\Gamma}{\FunApp{e}{}{\Many{e}}}{\tau}}
\]

Rules for expression typing: top level fields .f

Rules for expresion typing: type members \(\tau.f\).

Typing judgements for statements. \[
    \TypedOut{\Xi}{\Delta}{\Gamma}{s}{\tau}{\Gamma'}
\]

Typing judgements for declarations. \[
    \TypedOut{\Xi}{\Delta}{\Gamma}{D}{\tau}{\Gamma'; \Delta'}
\]

\newpage
Declarations.
\begin{Syntax}
    \SynClassAndCase{D}{\tau(\Many{e})\,x}
    \SynCase{\Package{T}{\Many{T}}{\Many{x : \tau\,p}}}
    \SynCase{\Control{T}{\Many{T}}{\Many{x : \tau\,p}}}
    \SynCase{\Control{T}{}{\Many{x : \tau\,p}}(\Many{x : \tau})\ \{
        \overline{D}\ \Keyword{apply}\ B\}}
\end{Syntax}

Types.
\begin{Syntax}
    \SynClassAndCase{\tau}{T}
    \SynCase{\Control{}{\Many{T}}{\Many{x : \tau\,p}}}
    \SynCase{\Parser{}{\Many{T}}{\Many{x : \tau\,p}}}
    \SynCase{\Package{}{\Many{T}}{\Many{x : \tau}}}
\end{Syntax}

Expressions.
\begin{Syntax}
    \SynClassAndCase{e}{x}
    \SynCase{\FunApp{\tau}{}{\Many{e}}}
\end{Syntax}


How should we typecheck the following program?
\begin{align*}
    &\Control{A}{}{}; \\
    &\Package{P}{}{a : A}; \\
    &\Control{C}{}{} \{ \}; \\
    &\Control{T}{}{}\ \{ \Keyword{apply}\ \{\}\} \\
    &A(C())\ x;
\end{align*}
Our judgment for checking declarations has the shape
\(\DoubleSequent{\Delta;\Gamma}{D}{\Delta';\Gamma'}\). We'll set it up so that
\(\Delta' = \Delta, D\) regardless of \(D\), and it should typecheck bodies of
controls. The last line (the instantiation) is the tricky part.

Rules for typechecking instantiations.
\[
\begin{array}{c}
    \inferrule[Inst-Control]{\Keyword{lookup}_\Delta(\tau) = \Control{T}{}{\Many{x
    : \tau\,p}}(x_1 : \tau_1, \dots, x_n : \tau_n)\ \{
        \overline{D}\ \Keyword{apply}\ B\}\\
    \Sequent{\Delta; \Gamma}{e_i: \tau_i} \text{ for each \(i\)}}
    {\Sequent{\Delta; \Gamma}{\tau(e_1, \dots, e_n) : \Control{}{}{\Many{x:\tau
    p}}}} \\
    \inferrule[Inst-Package]{\Keyword{lookup}_\Delta(\tau) = \Package{T}{}{x_1
    : \tau_1, \dots, x_n : \tau_n} \\
    \Sequent{\Delta; \Gamma}{e_i: \tau_i} \text{ for each \(i\)}}
    {\Sequent{\Delta; \Gamma}{\tau(e_1, \dots, e_n) : \Keyword{package}}}
\end{array}
\]

This isn't enough to typecheck the example. The type of the constructor argument
to \(P\) is \(A\), not \(\Control{}{}{}\). This can be dealt with by adding
a rule for "resolving" type names, but I think another option would be to inline
types. Maybe the way we do it in the formalization should be different from how
we do it in the implementation.

The rules do not handle generics.

\[
    \inferrule[Resolve-Control-Type]
    {\Sequent{\Delta; \Gamma}{e: \Control{}{\Many{T}}{\Many{x : \tau d}}} \\
     \Keyword{lookup}_\Delta(T_0) = \Control{T_0}{\Many{T}}{\Many{x : \tau d}}}
    {\Sequent{\Delta; \Gamma}{e: T_0}}
\]



\end{document}
